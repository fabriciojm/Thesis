\chapter*{Introducción}
\label{sec:introduction}

Una de las mayores búsquedas de la física del último centenar de años ha consistido en obtener un entendimiento de cómo el mundo funciona a la escala más pequeña, lo que ha llevado a notables descubrimientos. Con la ayuda de modelos matemáticos somos capaces de describir y predecir muy precisamente algunos fenómenos físicos observados, tales como el átomo y sus transiciones de energía, reacciones y decaimientos nucleares, por sólo nombrar un par de ellos.

La simple idea de hacer colisionar pequeñas partículas ha jugado un rol importante en esos descubrimientos y ha contribuido en gran medida al entendimiento de las leyes físicas a esa escala. Además, con la aparición de máquinas cada vez más poderosas, hemos sido capaces de observar con mejor detalle los componentes fundamentales de la naturaleza. Esto es análogo a la luz: su energía es inversamente proporcional a su longitud de onda, de manera que se requiere de mayor energía para resolver objetos más pequeños.

Los colisionadores de partículas se han convertido en los caballos de batalla de los físicos de partículas. Usando campos eléctricos y magnéticos, estas máquinas aceleran y curvan haces de partículas que luego colisionarán en lugares específicos, en donde están dispuestos detectores para observar el resultado de la interacción. Debido a que las partículas que chocan son muy pequeñas y las energías muy altas, el tratamiento matemático debe ser tanto mecánico-cuántico como relativista.

Dentro del actual marco de trabajo matemático-analítico (formalmente llamado el Modelo Estándar (ME)), sólo es posible manejar interacciones simples que involucran pocas partículas. Un tratamiento diferente es necesario para hacer los cálculos inherenetes a los procesos reales que ocurren en los colisionadores, donde típicamente cientos de partículas son creadas. Los programas de computadoras llamados ``generadores de eventos'' han venido siendo desarrollados para tratar tales situaciones de una manera fenomenológica, haciendo uso de modelos ajustables, inspirados tanto en predicciones teóricas como en resultados experimentales para ajustarse mejor a los datos. Además, los generadores han sido utilizados para explorar escenarios posibles, antes de que los experimentos reales sean montados y sus respectivos análisis llevados a cabo.

El estudio de una de las fuerzas fundamentales de la naturaleza, la interacción fuerte, es de particular interés en este trabajo. El campo ``gluónico'' proveniente de la interacción fuerte mantiene unidos a los quarks dentro de compuestos llamados hadrones, como los protones o neutrones. En las colisiones de altas energías, la interacción fuerte es la dominante, y a partir de ella (y de su campo gluónico) es posible crear pares de quarks de tipos más pesados que los que se encuentran en los protones y neutrones. El estudio de los hadrones que contienen estos quarks pesados puede ampliar nuestro conocimiento sobre cómo actúa dicha fuerza.

El generador de eventos \textsc{Pythia} (\cite{Sjostrand:2006za}, \cite{Sjostrand:2007gs}) será usado para analizar los diferentes mecanismos de producción de quarks y su rol en las colisiones. Varias modificaciones al algoritmo que usa PYTHIA para modelar la tasa de producción de pares $\g\to\Q\Qbar$ con $\Q$ un quark pesado (de tipo charm ($\c$) o bottom ($\b$)) han sido propuestas y serán probadas. La teoría subyacente es descrita en el capítulo \ref{sec:theoretical}.

El estudio de los mecanismos de producción comprende dos partes: colisiones electrón-positrón a energía de resonancia del $Z^0$ para comparar con datos del Gran Colisionador de Electrones y Positrones (LEP, por su sigla en inglés) y las más complicadas colisiones de hadrones, a energías típicas del Gran Colisionador de Hadrones (LHC). Varios observables físicos serán estudiados, como la separación angular de los objetos producidos a partir de los quarks pesados de las colisiones.

Gracias a que el generador provee una historia detallada del proceso simulado, se pueden trazar los orígenes de cada partícula y clasificarla. Los métodos para analizar los eventos son discutidos en el capítulo \ref{sec:analysis}. En algunos casos, los resultados (sección \ref{sec:results}) serán comparados con datos experimentales. Un resumen final y las perspectivas para futuros estudios se encuentran en el capítulo \ref{sec:summary}.


