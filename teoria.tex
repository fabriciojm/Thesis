\chapter{Resumen teórico}
\label{sec:theoretical}

El ME es la teoría física  que explica las propiedades de la materia y sus interacciones en su nivel más fundamental. Utilizando el marco de trabajo de la teoría especial de la relatividad y la mecánica cuántica, el ME lidia con las partículas elementales de la realidad física.

Las partículas pueden ser clasificadas de diferentes maneras, de acuerdo a sus propiedades, como el espín, la carga eléctria y la masa. Se puede hacer una primera distinción a partir de los valores de espín. Partículas con espín entero\footnote{Aquí y en el resto del trabajo, usaremos las unidades naturales: $\hbar$ (constante de Planck reducida) = $c$ (velocidad de la luz) = $e$ (carga del electrón) = 1.} son llamadas \textbf{bosones} y son las que median las interacciones, mientras que las que tienen espín semientero son llamadas \textbf{fermiones} y representan las partículas que interactúan. Por otra parte, para cada partícula del ME existe una antipartícula con las mismas propiedades pero cargas opuestas. Los fermiones son creados (y aniquilados) necesariamente en pares materia-antimateria, mientras que las interacciones son mediadas por los bosones, creados usualmente de manera individual.

Cuatro fuerzas fundamentales son actualmente conocidas en la naturaleza: las fuerzas nucleares fuerte y débil, el electromagnetismo y la gravedad. (Las primeras tres de ellas son descritas por el ME, mientras que la última aún no ha sido consistentemente incluida en dicho modelo.) La tabla \ref{table:SMBoson} lista las interacciones con sus respectivos bosones mediadores, es decir, las partículas asociadas al campo de la interacción.

\begin{table}[!h]
\caption{Interacciones en el Modelo Estándar y sus respectivos bosones.}\smallskip
\label{table:SMBoson}
\centering 

\begin{tabular}{cc}
\hline \hline  
\smallskip
Fuerza& Bosón (masa en GeV\cite{Beringer:1900zz}) \\ 
\hline
Electromagnetismo & $\gamma$ (0) \\
Interacción débil & $W^\pm$ y $Z^0$ ($80.4$ y $91.2$) \\
Interacción fuerte & $\g$ (0) \\
\end{tabular}
\end{table}


Además de la carga eléctrica, existen una carga para la fuerza fuerte llamada ``carga de color'', en analogía con el color de la vida cotidiana, como veremos luego. Los fotones ($\gamma$) y bosones $Z^0$ no tienen carga, de modo que son sus propias antipartículas, mientras que los gluones poseen sólo carga de color y los bosones $W^\pm$ tienen ($\pm 1$) carga eléctrica. Entonces, las antipartículas de los gluones son gluones con contenido de color opuesto, mientras que los bosones $W^+$ y $W^-$ son sus respectivas antipartículas.

Un bosón adicional no está incluido en la lista de la tabla 1, que es el bosón de Higgs (125 GeV), este bosón media el mecanismo de obtención de masa de las partículas. El universo está lleno del campo escalar neutro de Higgs, que tiene un valor de expectación no nulo en el vacío por el rompimiento espontáneo de la simetría del campo. Luego, las partículas interactúan con este campo y obtienen su masa. 

Hasta los momentos, 12 fermiones (y sus respectivas antipartículas) han sido observados, tal y como se muestran en la tabla 2. Cada columna en la tabla es llamada una \textbf{generación} o \textbf{familia}. Las segunda ($\c,\s,\mu,\nu_\mu$) y tercera ($\t,\b,\tau,\nu_\tau$) familias son ``versiones pesadas'' de la primera ($\u,\d,e,\nu_e$). Los neutrinos (en la última fila) son prácticamente no-masivos, los resultados experimentales han dado sólo un límite superior para sus masas. La carga eléctrica de cada quark en la primera fila es 2/3, mientras que los de la segunda fila tienen -1/3; los leptones de la tercera fila tienen carga -1 y sus neutrinos ($\nu$) son eléctricamente neutros.

A pesar de que todas las parículas hasta ahora mecionadas existen, los átomos están hechos únicamente de fermiones de la primera generación. El electromagnetismo mantiene a los electrones alrededor del núcleo, compuesto de neutrones y protones. Los últimos dos son combinaciones de quarks: ``udd'' y ``udd'' respectivamente. Protones, neutrones y otros objetos compuestos de tres (anti)quarks son conocidos como (anti)bariones, mientras que las combinaciones quark-antiquark dan lugar a mesones. Los bariones y mesones tienen el nombre genérico de hadrones.

Las partículas de la segunda y tercera familia son inestables y existen por lapsos cortos, decayendo luego a materia de la primera generación. Por ser más pesada, mayor energía es requerida para producir materia de este tipo; por eso es posible detectarla sólo en experimentos de altas energías, como colisionadores, y estudiando procesos cósmicos lo suficientemente energéticos.

Los quarks también interactúan a través de la fuerza fuerte, ya que tienen carga de color, a diferencia de los leptones (sin color). Por analogía con el color usual, existen tres tipos de carga básica involucradas en la interacción fuerte, llamados rojo ($r$), verde ($g$) y azul ($b$); y juntos pueden formar una combinación sin color (un singlete de color, en términos precisos) de la misma manera en que estos colores cotidianos sumados resultan en blanco. Esto significa que existen superposiciones mecánico-cuánticas que resultan en estados de singletes de color. Además, los anticolores son tales que se anulan cuando se combinan con su correspondiente color (por ejemplo, la combinación de rojo con anti-rojo ($r+\bar{r}$) debe ser neutral en color). Como dijimos anteriormente, los gluones también transportan carga de color, que transfieren durante las interacciones.

El contenido de color de los gluones es algo más complicado que el de los quarks, y esto se debe a la estructura de grupo de la teoría. Los gluones contienen un combinación no nula de color-anticolor. Teniendo 3 colores independientes, es natural tener los índices de color corriendo de 1 a 3 y una representación de matrices $3\times3$, como veremos en la siguiente sección. El grupo de simetría asociado con la fuerza fuerte es $SU(3)$, que tiene $3^2-1=8$ generadores linealmente independientes. Al sustraer uno, lo que estamos haciendo es excluir el generador que corresponde al estado de singlete. Así, los gluones son llamados ``octetos de color''.

Debido a que el color debe ser conservado, los gluones cargan una configuración no nula tal que puedan transmitir color y resultaren una nueva configuración ``coloreada'' de quarks o gluones luego de la interacción. Por ejemplo, el contenido de color de una emisión de un gluón por un quark rojo pudiera ser: $\q(r)\to\q(b)+\g(r\bar b)$.

Todos los hadrones en la naturaleza han sido observados como singletes de color. Así, existen varias maneras de producir tales estados, siendo las más comunes: la combinación de tres (anti)colores en un singlete para formar un (anti)barión, y la combinación de un quark con un antiquark en una combinación color-anticolor para formar mesones. La fuerza fuerte actúa sólo en objetos con carga de color; esto significa que la fuerza fuerte no es exclusivamente de los quarks, sino que también los gluones pueden interactuar con otros gluones. Este último proceso puede ser visto como sigue: nombrando a los gluones iniciales 1 y 4, luego 1 se divide en dos nuevos gluones, a los que llamaremos 2 y 3 ($\g(1)\to\g(2)+\g(3)$), el último de los cuales finalmente se une con 4 para formar un nuevo gluón, 5 ($\g(3)+\g(4)\to\g(5)$). Este proceso puede ser ilustrado por la figura \ref{fig:gluonGluon}, que es un ejemplo de los llamados diagramas de Feynman.

\begin{fmffile}{GluonGluon}

\begin{figure}[h]
  \centering
%(along, up)
    \begin{fmfgraph*}(150,100)
      \fmfstraight
      \fmfleft{i1,i2}
      \fmfright{o1,o2}
      \fmf{phantom}{i1,w1,w2,o1}
      \fmf{gluon,label=$\g(1)$}{i1,w1}
      \fmf{gluon,label=$\g(2)$}{w1,o1}
      \fmf{phantom}{i2,w3,w4,o2}
      \fmf{gluon,label=$\g(4)$}{i2,w4}
      \fmf{gluon,label=$\g(5)$}{w4,o2}
      \fmf{gluon,label=$\g(3)$}{w1,w4}
    \end{fmfgraph*}
\caption[Gluon-gluon interaction]{Diagrama de Feynman para una interacción gluón-gluón.}
\label{fig:gluonGluon}
\end{figure}

\end{fmffile}

Estos diagramas representan las interacciones del ME. Las partículas son dibujadas como líneas y las interacciones son los vértices. Los fermiones son representados usualmente por líneas rectas sólidas, mientras que los bosones por líneas rizadas (gluones), onduladas ($W, Z, \gamma$) o punteadas (Higgs). Tal y como sugiere el párrafo anterior, el tiempo fluye de izquierda a derecha en el diagrama. A pesar de que los diagramas de Feynman no representan las trayectorias reales de las partículas, estos constituyen una herramienta poderosa para describir la naturaleza de las interacciones y hacer cálculos de elementos de matriz, que serán descritos más tarde.

Dos características relevantes de la fuerza fuerte son:

\paragraph{Confinamiento} Debido a que los quarks individualmente no pueden formar un singlete de color, no se ha observado hasta ahora ningún quark libre. A distancias mayores a $10^{-15}$ m, se cree que la fuerza fuerte por intercambio de gluones es constante, así que la energía almacenada entre quarks crece linealmente con la distancia cuando se intenta separarlos para descomponer un hadrón. Para ilustraar esto, tomaremos el ejemplo simple de un mesón: cuando se da al sistema suficiente energía para separar a los dos quarks, el campo gluónico creará un nuevo par quark-antiquark entre ellos. De esta manera, la interacción original será apantallada, formando dos nuevos mesones, es decir, cada quark de los extremos interactuará sólo con su vecino más cercano del nuevo par. Entonces este proceso asegura que todos los quarks permanecerán confinados en hadrones.

\paragraph{Libertad asintótica} La fuerza fuerte cambia su comportamiento a distancias muy pequeñas. En ese caso, ésta es menos fuerte y los quarks interactúan de una manera más débil. Los quarks tienden entonces asintóticamente a ser objetos libres. De ese modo, la teoría puede ser tratada perturbativamente, como veremos en la sección \ref{subsec:QCDLag}.

\section{El Lagrangiano del Modelo Estándar y los elementos de matriz}
\label{subsec:QCDLag}

La parte del modelo estándar que trata la interacción fuerte es llamada Cromodinámica Cuántica (QCD, por su sigla en inglés). La intensidad y las reglas de las interacciones están explícitamente expuestas en el lagrangiano. Los procesos en los que estamos interesados son los que involucran interacciones entre quarks y gluones. La parte del lagrangiano para estas interacciones tiene la forma \cite{Kane:1993}:

 \begin{equation}
  \frac{g_s}2 \qbar_\alpha \gamma^\mu \lambda^a_{\alpha \beta} G_\mu^a \q_\beta\label{eq:QCDLag}
  \, .
\end{equation}

En la fórmula, $g_s$ es la intensidad del acoplamiento de la fuerza fuerte. Tanto $\q_\beta$ como $\qbar_\alpha$ son estados espinoriales de los quarks con sus respectivos índices de color ($\alpha,\beta = 1,2,3$), $\gamma^\mu$ representa una matriz de Dirac y $\lambda^a_{\alpha\beta}$ la matriz de color de Gell-Mann, que son los generadores del grupo SU(3). $G_\mu^a$ es la intensidad del campo gluónico. El índice $\mu$ es espacio-temporal (0,1,2,3), mientras que el índice del campo gluónico $a=1,...,8$. El lagrangiano completo del ME también comprende los términos de masa y las autointeracciones de los gluones.

Físicamente, la expresión (\ref{eq:QCDLag}) representa los siguiente mecanismos: emisión de un gluón por un quark ($\q\to\g\q$), absorción de un gluón por un quark ($\g\q\to\q$) y creación de un par a partir de un gluón ($\g\to\qbar\q$); además de los correpondientes procesos invertidos en el tiempo: absorción de un gluón por un antiquark ($\g\qbar\to\qbar$), emisión de un gluón por un antiquark ($\qbar\to\g\qbar$) y aniquilación de un par en un gluón ($\q\qbar\to\g$).

\begin{fmffile}{feyndiag}

\begin{figure}[h]
  \centering
%(along, up)
    \begin{fmfgraph*}(100,100)
      \fmfleft{i1}
      \fmfright{o1,o2}
      \fmf{fermion,label=$\q$}{i1,w1}
      \fmf{fermion,label=$\q$}{w1,o1}
      \fmf{gluon,label=$\g$}{o2,w1}
    \end{fmfgraph*}
    \hspace{2em}
    \begin{fmfgraph*}(100,100)
      \fmfleft{i1,i2}
      \fmfright{o1}
      \fmf{fermion,label=$\q$}{i1,w1}
      \fmf{gluon,label=$\g$}{i2,w1}
      \fmf{fermion,label=$\q$}{w1,o1}
    \end{fmfgraph*}
    \hspace{2em}
    \begin{fmfgraph*}(100,100)
      \fmfleft{i1}
      \fmfright{o1,o2}
      \fmf{gluon,label=$\g$}{w1,i1}
      \fmf{fermion,label=$\q$}{w1,o1}
      \fmf{fermion,label=$\qbar$}{o2,w1}
    \end{fmfgraph*} \\
    \vspace{3em}
        \begin{fmfgraph*}(100,100)
      \fmfleft{i1,i2}
      \fmfright{o1}
      \fmf{fermion,label=$\qbar$}{w1,i1}
      \fmf{gluon,label=$\g$}{i2,w1}
      \fmf{fermion,label=$\qbar$}{o1,w1}
    \end{fmfgraph*}
    \hspace{2em}
    \begin{fmfgraph*}(100,100)
      \fmfleft{i1}
      \fmfright{o1,o2}
      \fmf{fermion,label=$\qbar$}{w1,i1}
      \fmf{fermion,label=$\qbar$}{o1,w1}
      \fmf{gluon,label=$\g$}{o2,w1}
    \end{fmfgraph*}
    \hspace{2em}
    \begin{fmfgraph*}(100,100)
      \fmfleft{i1,i2}
      \fmfright{o1}
      \fmf{fermion,label=$\qbar$}{w1,i1}
      \fmf{fermion,label=$\q$}{i2,w1}
      \fmf{gluon,label=$\g$}{o1,w1}
    \end{fmfgraph*}
\caption[QCD quark diagrams]{Diagramas de Feynman para los procesos representados por la ec. (\ref{eq:QCDLag}). }
\label{fig:feynDiag}
\end{figure}

\end{fmffile}

Los diagramas de Feynman para los procesos mencionados están ilustrados en la figura \ref{fig:feynDiag}. Las flechas en las líneas de quarks representan el flujo fermiónico y un antiquark es entonces representado como un fermión que viaja hacia el pasado.

Una de las cantidades más importantes involucradas en los cálculos del ME es la amplitud de probabilidad de las interacciones, la cual está relacionada con observables físicos, tales como secciones eficaces y tasas de decaimiento. La mecánica cuántica establece que las probabilidades de medición son elementos de matriz al cuadrado $|M|^2$, que vienen de la proyección del estado final medido sobre el estado que resulta de la evolución temporal del estado inicial. En un lenguaje matemático, la amplitud $M$ es calculada como:
\begin{equation*}
M=\Bra{\text{final}}S\Ket{\text{inicial}}, 
\end{equation*}
donde $S$ es el operador evolución, que está íntimamente relacionado con el lagrangiano. En el ME, esta evolución es llevada a cabo de una manera perturbativa. Cada término en la expansión perturbativa es asociado con un grafo de Feynman que contribuye a la amplitud final: el término de más bajo orden corresponde a un diagrama de ``árbol'', mientras que los términos de órdenes más altos corresponden a diagramas de ``lazo'' o a estados finales de mayor multiplicidad. Un ejemplo de diagrama de árbol y uno de lazo se muestran en la figura \ref{fig:Loop}.

\begin{fmffile}{Loop}%

\begin{figure}[!h]
  \centering
%(along, up)
    \begin{fmfgraph*}(150,150)
      \fmfleft{i1,i2}
      \fmfright{o1,o2}
      \fmf{quark,label=$\q$}{i1,w1}
      \fmf{quark,label=$\qbar$}{w1,i2}
      \fmf{gluon,label=$\g$}{w1,w2}
      \fmf{quark,label=$\q$}{w2,o1}
      \fmf{quark,label=$\qbar$}{o2,w2}      
    \end{fmfgraph*}
    \hspace{2em}
     \begin{fmfgraph*}(150,150)
      \fmfleft{i1,i2}
      \fmfright{o1,o2}
      \fmf{quark,label=$\q$}{i1,w1}
      \fmf{quark,label=$\qbar$}{w1,i2}
      \fmf{gluon,label=$\g$}{w1,v1}
      \fmf{quark,label=$\q$}{w2,o1}
      \fmf{quark,label=$\qbar$}{o2,w2}
      \fmf{phantom,tension=1}{w1,v1}
       \fmf{phantom,tension=1}{v2,w2}
       \fmf{fermion,left,tension=0.4}{v1,v2,v1}
      \fmf{gluon,label=$\g$}{v2,w2}
    \end{fmfgraph*}
  \vspace{1em}
\caption[Diagramas de árbol y de lazo.]{Diagramas de árbol (izquierda) y de lazo (derecha) que contribuyen al mismo proceso.}
\label{fig:Loop}
\end{figure}

\end{fmffile}


Cuando las amplitudes son calculadas para los diagramas, las correcciones dadas por los términos de más alto orden dependen de la transferencia de momentum de la interacción. Entonces, la intensidad de la interacción ($g_s$) debe incluir las contribuciones de todos los posibles grafos, conllevando a una dependencia entre el acoplo y la transferencia de momentum $Q$.

La dependencia del acoplo de la interacción fuerte con la transferencia de momentum está dada por
\begin{equation}
\as(Q^2) = \frac{12 \pi}{(33-2n_f)\ln(Q^2/\Lambda^2)},
\label{eq:alphastrong}
\end{equation}
donde $\alpha_s=g_s/(4\pi)$, $n_f$ es el número de tipos (sabores) de quarks y $\Lambda$ es la escala de energía de la QCD. Luego, el acoplo de la fuerza fuerte tiene una divergencia logarítmica cuando $Q^2$ está cerca de $0.3$ GeV, que es el valor estimado de $\Lambda$. De esto sigue que la QCD perturbativa funciona adecuadamente cuando la expansión es tomada en una región donde el valor de $\alpha_s$ no es demasiado grande, es decir, a energías mucho mayores que 1 GeV. La dependencia del acoplo con la tranferencia de momentum explica porqué el confinamiento domina a escalas de $Q^2$ pequeño, cuando $\alpha_s$ diverge, mientras que la libertad asintótica aparece a valores altos de $Q^2$, donde $\alpha_s$ tiende a cero.

La complejidad de cálculo de los elementos de matriz escala factorialmente con el número de partículas del estado final \cite{Peskin:1995}. Así que este método es manejable cuando se tratan pocas partículas, pero no en un colisionador como el LHC, donde típicamente se producen alrededor de cien partículas por evento, lo que conlleva a expresiones matemáticas fuera de nuestra capacidad de cálculo. Una complicación adicional es que muchas de las partículas detectadas en el estado final son hadrones y el tratamiento perturbativo de QCD trata únicamente quarks y gluones.

\section{El generador de eventos \textsc{Pythia} y la lluvia de partones}

Los generadores de eventos pueden ayudar a superar las dificultades de usar únicamente elementos de matriz. El generador de eventos \textsc{Pythia} usa métodos de Monte Carlo para simular los eventos que toman lugar en colisionadores.

Desde el punto de vista mecánico-cuántico, un sistema cambia su estado de acuerdo con la probabilidad de que el siguiente estado ocurra. Como dijimos anteriormente, esta probabilidad está relacionada con la amplitud (elemento de matriz) al cuadrado. Entonces, el mismo experimento (por ejemplo, una colisión de hadrones) puede producir diferentes estados finales, de modo que los resultados adquieren un valor significativo una vez que se han realizado suficientes observaciones. Los generadores de eventos de Monte Carlo usan números (seudo)aleatorios para emular la ``elección'' cuántica del estado subsiguiente en la evolución.

Los constituyentes de los hadrones que colisionan son conocidos como partones. Luego, el proceso duro es definido como la subcolisión de partones $2\to 2$ más energética de los hadrones de entrada. Debido a que los protones están constituidos por quarks de valencia (uud), gluones y el mar de quarks ($\u\ubar, \d\dbar, \c\cbar,\dots$), el proceso duro puede empezar con cualquier par de estos objetos, uno de cada hadrón que entra a la colisión.

Los elementos de matriz son utilizados para los cálculos en el proceso duro y la evolución subsecuente de las partículas es modelada por un algoritmo de ``lluvia de partones'''. La idea es modelar las ramificaciones permitidas por la interacción fuerte: emisión de un gluón por un (anti)quark ($\q\to\q g$), ramificación de un gluón en dos gluones (($\g\to\g\g$) y creación de un par quark-antiquark a partir de un gluón ($\g\to\q\qbar$). El proceso menos probable, de ramificación de un gluón en tres gluones ($\g\to\g\g\g$) no está incluido en la lluvia de partones.

Despreciando las masas de los quarks, la probabilidad de una ramificación en el límite
colineal está gobernada por las ecuaciones DGLAP (ver \cite{Sjostrand:2009ad}):

\begin{equation}
  \d P_{a\to bc} = \frac{\as}{2\pi}\frac{\d Q^2}{Q^2}\mathcal{P}_{a\rightarrow bc}(z) \d z\label{eq:DGLAP}
  \, ,
\end{equation}
donde

$$
\mathcal{P}_{\q\to \q\g} = \frac43 \frac{1+z^2}{1-z}
  \, , \hspace{2em}
\mathcal{P}_{\g\to \g\g} = 3 \frac{(1-z(1-z))^2}{z(1-z)}
  \, , \hspace{2em}
\mathcal{P}_{\g\to \q\qbar} = \frac{n_f}2 (z^2+(1-z)^2).
$$

La variable $z$ representa la compartición de la energía luego de la ramificación: $E_b=zE_a$ y $E_c=(1-z)E_a$. $Q$ es la virtualidad del proceso (es decir, del quark o gluón inicial). Este procedimiento es aplicado recursivamente para obtener las ramificaciones sucesivas de las partículas hijas.

Las ecuaciones DGLAP son confiables en el caso de emisiones fuertemente ordenadas, es decir, cuando la virtualidad de la partícula hija es mucho menor que la de la madre. Ese es el caso de la radiación de las partículas que emergen del proceso duro, conocidas como Radiación del Estado Final (FSR). Gracias a que las partículas involucradas tienen virtualidades positivas, esta radiación también es conocida como lluvias de partones tipo tiempo. Del principio de Heisenberg sigue que no tenemos una total certeza en el ordenamiento temporal de la lluvia; sin embargo, supondremos que las emisiones subsecuentes conllevan a virtualidades menores.

En una colisión de hadrones, la radiación a partir de los partones antes de la colisión lleva el nombre de Radiación del Estado Incial (ISR). En contraste con la FSR, estas lluvias tipo espacio (de virtualidades negativas) son más complicadas, debido a la incertidumbre en la estructura de los partones de entrada: cómo el momentum está distribuido entre los partones, creación y aniquilación de quarks en el mar, etc. En este caso, la construcción de la lluvia es llevada a cabo usando una ``evolución inversa''; esto es, una vez que el proceso duro ha sido seleccionado, se disminuye hacia virtualidades más bajas hasta alcanzar el estado inicial.

Las ecuaciones DGLAP son singulares en el régimen ``colineal'' ($Q\to0$) y dos de ellas (en los casos $\q\to\g\q$ y $\g\to\g\g$) también lo son para para emisiones ``suaves'', cuando la energía se comparte en los casos extremos $z\to1$ y $z\to0$. Para evadir ambas singularidades, se introduce un corte inferior alrededor de 1 GeV, donde el confinamiento se vuelve dominante.

Ahora estudiaremos el caso del decaimiento radiactivo, para usarlo como analogía con la ramificación de partones. Denotemos el número de núcleos radiactivos que no han decaído en un tiempo $t$ por $\mathcal N(t)$ y el número inicial (en $t=0$) por $\mathcal N_0$. Un primer ansatz sería $\d\mathcal N(t)/\d t = -c \mathcal N_0$, donde $c$ es un parámetro constante que representa la la probabilidad de decaimiento por unidad de tiempo. La solución es entonces $\mathcal N (t) = \mathcal N_0(1-ct)$, lo cual no puede ser cierto, ya que para tiempos $t>1/c$, el número de núcleos que no han decaído se torna negativo, y la probabilidad de haber tenido un decaimiento excede la unidad. La manera de reparar tal defecto es poponiendo un nuevo ansatz donde la razón de decaimiento toma en cuenta el efecto de tener núcleos que ya decayeron. Esto es llevado a cabo introduciendo $\d\mathcal N(t)/\d t = -c \mathcal N(t)$; luego la solución es $\mathcal N(t) = \mathcal N_0 \exp(-ct)$. Este decaimiento exponencial encaja bien en el modelo, ya que el número de núcleos que no ha decaído tiende a cero y la probabilidad a la unidad cuando $t\to \infty$. En el caso que $c$ fuera una función del tiempo, $c=c(t)$, la probabilidad de decaimiento a un tiempo dado $t$ es modificada a
$$P(t)=-\frac{1}{\mathcal N_0} \frac{\d\mathcal N (t)}{\d t} = c(t) \exp\left(-\int_0^t c(t') \d t'\right).$$
La probabilidad de que un núcleo no haya decaído a un tiempo $t$ es entonces
$$1- \int_0^t P(t')\d t' = \exp \left(-\int_0^t c(t')\d t' \right),$$
que tiende a cero para valores grandes de $t$, (a menos que $c(t)$ se anule para $\t\to\infty$).

La probabilidad que habíamos escrito para la ramificación de partones (ecuación \ref{eq:DGLAP}) debe entonces ser modificada para incluir la atenuación exponencial que hace que se conserve la probabilidad. Las ecuaciones DGLAP son modificadas correspondientemente por el llamado factor de forma de Sudakov:
\begin{equation}
  \d P_{a\to bc} = \frac{\as}{2\pi}\frac{\d Q^2}{Q^2}\mathcal{P}_{a\to bc}(z) \d z \exp \left( -\sum_{b,c}\int^{Q^2_{max}}_{Q^2}\frac{\d Q'^2}{Q'^2} \int \frac{\as}{2\pi} \mathcal{P}_{a\to bc}(z') \d z' \right)
  \label{eq:Sudakov}
  \, .
\end{equation}

El principio de incertidumbre nos dice que la escala de tiempo relevante para la ramificación va como $\Delta t \sim 1/\Delta E \sim 1/\Delta Q$, así que la evolución temporal es llevada a cabo por integración en $Q^2$, empezando con su valor máximo y bajando hasta que el corte es alcanzado. Es posible hacer que esta evolución corra en otras variables, como el momentum transverso de la ramificación $\pT^2$, o el ángulo de emisión $\theta^2$, siempre usando el jacobiano apropiado. Además, el exponente en la ecuación (\ref{eq:Sudakov}) suma sobre todas las posibles partículas ramificadas.

\section{La tasa $\g\to\Q\Qbar$}

Ahora enfocaremos nuestra atención en el caso de quarks masivos para modelar el proceso en el que estamos interesados: un gluón creando un par pesado quark-antiquark.

\subsection{El kernel DGLAP}

La ecuación DGLAP para la creación de pares debe ser modificada para incluir el efecto no despreciable de la masa. La expresión estándar (en términos de la escala de evolución dada por la masa invariante $m^2=Q^2$)
\begin{equation}
\d P_{\g \to \q\qbar} = \frac{\as}{2\pi} \, \frac{\d m^2}{m^2} \,
\frac{1}{2} \left( z^2 + (1 - z)^2 \right) \, \d z ~ ,
\label{eq:DGLAP:gqq}
\end{equation}
deja de ser válida. Introduciendo
\begin{eqnarray}
 r_{\Q} & = & \frac{ m_{\Q}^2 }{ m^2 } ~, \\
\beta_{\Q} & = & \sqrt{ 1 - \frac{ 4 m_{\Q}^2 }{ m^2 } }
   = \sqrt{ 1 - 4 r_{\Q} } ~,
\end{eqnarray}
donde $\beta_Q$ es la magnitud de la velocidad de cada quark pesado en el sistema de referencia en el que el par está en reposo, la tasa de producción de pares de DGLAP es modificada a
\begin{equation}
\d P_{\g \to \Q\Qbar} = \frac{\as}{2\pi} \, \frac{\d m^2}{m^2} \,
\frac{\beta_{\Q}}{2} \left( z^2 + (1 - z)^2 + 8 r_{\Q} z (1 - z) \right) 
\, \d z ~ , 
\end{equation}
y luego de integrarlo en $z$
\begin{equation}
\frac{\d P_{\g \to \Q\Qbar}}{\d m^2} = \frac{\as}{2\pi} \, \frac{1}{m^2} \,
\frac{1}{3} \beta_{\Q} (1 + 2 r_{\Q}).
\label{m2DGLAP} 
\end{equation}
Nos referiremos a esta expresión como el resultado DGLAP.

\subsection{Una expresión de elemento de matriz}

Ahora pondremos la ramificación ($\g\to\Q\Qbar$) en el contexto de un proceso real, para ver su efecto en una expresión de elemento de matriz. Gracias a que el bosón de Higgs es una partícula escalar y no carga color, sus decaimientos son isotrópicos, lo cual es conveniente para hacer integraciones. Una manera limpia de colocar el proceso en el que estamos interesados a partir del bosón de Higgs, es tomando el decaimiento $\H\to\g\g$ \footnote{Este decaimiento ocurre a través de un lazo de quark top; esto es omitido acá y en el resto de la discusión.} e incluyendo un decaimiento sucesivo de uno de los gluones a dos quarks pesados $\H\to\g\g\to\g\Q\Qbar$. En la figura \ref{fig:Higgs} podemos ver diagramas de Feynman para ambos decaimientos. 

\begin{fmffile}{Higgs}%

\begin{figure}
  \centering
%(along, up)
    \begin{fmfgraph*}(150,150)
      \fmfleft{i1}
      \fmfright{o1,o2,o3,o4}
      \fmf{dashes,label=$\H$}{i1,w1}
      \fmf{gluon,label=$\g$}{w1,w2}
      \fmf{gluon,label=$\g$}{w3,w1}
      \fmf{phantom}{o1,w2}
      \fmf{phantom}{w2,o2}
      \fmf{phantom}{w3,o3}
      \fmf{phantom}{w3,o4}
    \end{fmfgraph*}
    \hspace{2em}
     \begin{fmfgraph*}(150,150)
      \fmfleft{i1}
      \fmfright{o1,o2,o3,o4}
      \fmf{dashes,label=$\H$}{i1,w1}
      \fmf{gluon,label=$\g$}{w1,w2}
      \fmf{gluon,label=$\g$}{w3,w1}
      \fmf{quark,label=$\Q$}{w2,o1}
      \fmf{quark,label=$\Qbar$}{o2,w2}
      \fmfv{lab=(3),lab.dist=0.005w}{w3}
      \fmfv{lab=(1),lab.dist=0.005w}{o1}
      \fmfv{lab=(2),lab.dist=0.005w}{o2}
      \fmf{phantom}{w3,o3}
      \fmf{phantom}{w3,o4}
    \end{fmfgraph*}
  \vspace{1em}
\caption[Decaimiento del Higgs a dos y tres cuerpos.]{Decaimiento del bosón de Higgs a dos (izquierda) y tres (derecha) cuerpos.}
\label{fig:Higgs}
\end{figure}

\end{fmffile}

Para obtener la expresión del elemento de matriz que nos interesa, calcularemos la cantidad $\d\Gamma_3/\Gamma_2$, es decir, que estudiaremos la variación de la tasa de decaimiento cuando uno de los gluones se separa en un par de quarks pesados, normalizado con respecto al primer decaimiento (a solo dos cuerpos):
\begin{equation}
\frac{\d \Gamma_3}{\Gamma_2} 
 = \frac{\as}{2\pi} \, 
\left( \frac{x_1^2 + x_2^2}{1 - x_3} - 2 + 2r \frac{x_3^2}{(1 - x_3)^2} 
\right) \, \d x_1  \, \d x_2 ~,
\end{equation}
donde estamos usando la fracción de energía $x_i=2E_i/M=2p_0p_i/M^2$ (con $i$ el número de la partícula en la figura \ref{fig:Higgs}), $r=m_Q^2/M^2$ y $\Gamma_2, \Gamma_3$ son las tasas de decaimiento en dos y tres cuerpos, con $M=m_H$. Haciendo algunas manipulaciones algebráicas a esta ecuación, tomando el límite $m\to 0$ y suponiendo quarks no masivos, se recupera la expresión original de DGLAP.

Podemos relacionar las fracciones $x_{1,2}$ al $\cos\theta$ del par en el sistema de referencia de reposo del gluón, tomado en el plano $xz$:
\begin{equation}
p_{\Q,\Qbar} = \frac{m}{2} \left( 1; \pm \beta_{\Q} \sin\theta, 0, 
 \pm \beta_{\Q} \cos\theta \right) ~.
\label{kinp}
\end{equation}
Usando la proporción $\delta=m^2/M^2$, un boost de Lorentz en el eje $z$ es $\beta_z=(1-\delta)/(1+\delta)$. Luego del boost, los momenta son
\begin{eqnarray}
x_{1,2} & = & \frac{1}{2} \left(1 + \delta \pm (1 - \delta)
\, \beta_{\Q} \, \cos\theta \right) ~, \label{kinx} \\
x_3 & = & 1 - \delta ~.
\end{eqnarray}
Para la integración, usaremos que $r/\delta=(m_\Q^2/M^2)/(m^2/M^2)=m_\Q^2/m^2=r_\Q$, así que la expresión de elemento de matriz integrada se convierte en
\begin{eqnarray}
 &  & \int_{x_{1,\mrm{min}}}^{x_{1,\mrm{max}}} \left( x_1^2 + x_2^2 - 2(1 - x_3) 
+ 2r \frac{x_3^2}{(1 - x_3)} \right)  \, \d x_1 \nonumber \\
& = & \int_{-1}^1 \frac{1}{2} \, \left( (1 + \delta)^2 
+ (1 - \delta)^2 \, \beta_{\Q}^2 \, \cos^2\theta - 4 \delta 
+ 4 \, \frac{r}{\delta} \, (1 - \delta)^2 \right) \, \frac{1}{2} 
\, (1 - \delta) \, \beta_{\Q} \, \d(\cos\theta) \nonumber \\
& = & \frac{2}{3} \, \beta_{\Q} \, (1 + 2 r_{\Q}) \, (1 - \delta)^3 ~, 
\label{m2ME} 
\end{eqnarray}
es decir, recuperamos de nuevo la expresión de DGLAP, pero con una supresión dada por el factor $(1-\delta)^3$ para valores grandes de la masa invariante del gluón, y un factor de dos, por tener dos patas de gluón. Nos referiremos a este resultado como la expresión de elemento de matriz (EM).

Las ecuaciones DGLAP no aseveran incluir los efectos de los factores de espacio de fase tomados en cuenta en la integración que acabamos de llevar a cabo. A pesar de que una supresión dada por $(1-\delta)^3$ es plausible, no hay garantía de que este factor sea universal a todos los procesos.

\subsection{El algoritmo de \textsc{Pythia}}
\label{subsubsec:PythiaAlg}

Ahora describiremos algunos detalles del algoritmo de \textsc{Pythia} que involucran el decaimiento de un gluón en un par de quarks. Luego de eso, las opciones a probar con sus respectivos pesos de decaimiento son descritos.

La evolución en el algoritmo está fijada para ser en términos de un $\pTse$ decreciente. Esta nueva variable es cercana al $\pTs$ normal, pero tiene algunas ventajas para separaciones angulares grandes. Dado este valor, el rango de $z$ permitido es
\begin{equation}
z_{\mrm{max,min}}(\pTse) = \frac{1}{2} \pm \sqrt{ \frac{1}{4}
 - \frac{\pTse}{M^2}},
\end{equation}
donde $M$ es ahora la masa del dipolo que forman el par de quarks.

Debido a que $z^2+(1-z)^2<1$, la tasa de la evolución puede ser sobreestimada por la longitud del rango máximo permitido de $z$ (dado por el corte inferior de $\pTse$), multiplicado por un medio para cada sabor de quarks. Aquí, los valores de $z$ son tomados planos; esta sobreestimación será corregida. Para la evolución subsecuente en $\pTse$, los decaimientos potenciales con valores de $z$ fuera del rango permitido son rechazados.

Para un par consistente $(\pTse,z)$, calculamos el valor de la masa para el par de quarks:
\begin{equation}
m^2 = \frac{\pTse}{z(1-z)}.
\end{equation}
El jacobiano para esta transformación tiene la conveniente propiedad que
\begin{equation}
\frac{\d\pTse}{\pTse} \, \d z = \frac{\d m^2}{m^2} \, \d z,
\label{jacobian}
\end{equation}
así que el espacio de fase puede ser cubierto por cualquiera de las variables de evolución. La preferecia de una sobre otra depende del efecto que ellas tienen en el factor de Sudakov. Debido a que cada variable cubre el espacio de fase de diferente forma, puede que algunos lugares sean alcanzados en diferentes escalas en estas variables. Esto significa que la integral en el factor de Sudakov va a diferir.

A continuación se describen las opciones a estudiar, que involucran la tasa de decaimiento de $\g\to\Q\Qbar$.

\paragraph{Opción 1}

Esta es la opción por defecto, que ha sido implementada hasta ahora. Parte de un par de quarks (de cualquier sabor), y el efecto de las masas es introducido a posteriori, ``encogiendo'' el momentum espacial, pero manteniendo los ángulos de decaimiento fijos. Esto es llevado a cabo cambiando al sistema de referencia en reposo del gluón, donde el par es producido con las direcciones de los quarks opuestas, e incrementando las masas de éstos de acuerdo con el sabor. El 3-momentum de cada quark es respectivamente reducido en magnitud y los ángulos son preservados. El peso para la probabilidad de supervivencia del decaimiento es asignado a $W=\beta_\Q(z^2+(1-z)^2)$. Debajo del umbral de masa para la creación del par de quarks, $\beta_Q=0$. El valor de este peso es menor al dado por las expresiones de ME y DGLAP para la región de umbral, ya que no contiene un término dependiente de la masa.

\paragraph{Opción 2}

Una corrección directa a la Opción 1 es añadir a $W$ el término de masa que falta. Esta corrección es válida debido a que el nuevo peso sigue siendo menor que la unidad para todo $z$. Esta opción tiene el mismo comportamiento a bajas masas que DGLAP y que EM.

En el límite $m^2\to M^2$, esta opción de Pythia cae más rápido que el resultado dado por DGLAP, pero no tan rápido como la expresión de elemento de matriz, debido a que este último tiene un factor de supresión fuerte, que no es necesariamente universal.

\paragraph{Opción 3}

Esta opción reproducirá el comportamiento de DGLAP para el rango completo de $m^2$. Esperameos que dicho comportamiento sea similar a EM a masas bajas pero también hay que observar el otro límite. Siguiendo la expresión para calcular las masas es fácil observar que un dado valor de $m^2$ puede ser alcanzado a partir de diferentes valores de $\pTse$ y $z$. Usaremos la cinemática de las ecuaciones (\ref{kinp}) y (\ref{kinx}) para obtener una expresión para el peso. Debido a que el punto de partida es el par de quarks no masivos, $\beta_\Q=1$; además, tomando en cuenta que $z=x_1/(x_1+x_2)$, el rango permitido va desde $\cos\theta=-1$ a 1, es decir, de $m^2/(M^2+m^2)$ a $M^2/(M^2+m^2)$. Notando que $z(m^2)$ es también plana (tal y como $z(\pTse)$, debido al jacobiano) e introduciendo

\begin{equation}
I_z(m^2) = \int_{z_{\mrm{min}}(m^2)}^{1 - z_{\mrm{min}}(m^2)} 
\left( z^2 + (1 - z)^2 + 8 r_{\Q} z (1 - z) \right) \, \d z,
\label{intz}
\end{equation}
se sigue que el nuevo peso

\begin{equation}
W = \frac{2}{3} \, \beta_{\Q} \, (1 + 2 r_{\Q}) \,
\frac{ z^2 + (1 - z)^2 + 8 r_{\Q} z (1 - z) }{I_z(m^2)}
\label{wtone}
\end{equation}
promediará en $(2/3)\beta_\Q(1+2r_\Q)$. Recobramos entonces la tasa de DGLAP de la ecuación (\ref{m2DGLAP}), ya que la sobreestimación hecha en el algoritmo tiene un factor de 1/2. Sin embargo, este enfoque tiene problemas con el límite $m^2\to M^2$, donde el rango permitido de $z$ se restringe al valor de 1/2. Esto conlleva a un peso casi constante, que no obedece a la esperada dependencia angular $1+\cos^2\theta (\sim z^2+(1-z)^2)$. Para resolver esto, introduciremos

\begin{equation}
z_{\theta} = \frac{1 + \cos\theta}{2} = 
\frac{ (1 + \delta) \, z - \delta}{1 - \delta}
\end{equation}Luego, el análogo a $I_z(m^2)$ con esta nueva variable será $(2/3)(1+2r_\Q)$ (como en la integración para obtener la ec. (\ref{m2DGLAP})), multiplicado por el jacobiano que relaciona a $z$ con $z_\theta$, es decir,
\begin{equation}
W = \beta_{\Q} \, \left( z_{\theta}^2 + (1 - z_{\theta})^2 + 8 r_{\Q} z_{\theta} 
(1 - z_{\theta}) \right) \, \frac{1 + \delta}{1 - \delta} ~.
\label{wttwo}
\end{equation}
El peso en esta ecuación define la Opción 3. Gracias a que el denominador $1-\delta$, la la tasa de decaimiento dada por DGLAP decae más más lentamente que la implementada hasta ahora en \textsc{Pythia}, en la ec. (\ref{wttwo}). Este peso diverge en el límite $m^2\to M^2$, así que una manera de repararlo es aumentando la tasa del decaimiento de prueba $\g\to\q\qbar$ por un factor (fijado a 20) que es luego usado para disminuir $W$ correspondientemente.

\paragraph{Opción 4}

Ahora, se desea reproducir el comportamiento de ME. Para recuperar la ec. (\ref{m2ME}) se necesita multiplicar el peso de la Opción 3 por el factor de supresión $(1-\delta)^3$. La supresión tiene un efecto más fuerte mientras la masa sea mayor.

Para resumir, las cuatro opciones son:

\begin{enumerate}
\item La opción por defecto usada hasta ahora, creando quarks no masivos y agregando las masas a través de la disminución del 3-momentum en el sistema en reposo. El valor del peso en esta opción es menor que la tasa de DGLAP y EM en la región del umbral porque no contiene un término que agrega la dependencia de la masa. A masas altas, cae más rápidamente que DGLAP pero más lentamente que EM debido a la fuerte supresión en el último.
\item Una corrección inmediata: añade el término de masa que soluciona el problema en la zona de umbral. En esta región, el comportamiento es similar a DGLAP y EM.
\item Reproduce la forma de DGLAP para todo el rango de masas. Incrementa para valores altos de $m^2(\to M^2)$, así que es probable que no reproduzca los valores deseados a esa escala.
\item Para recuperar el comportamiento de EM, el peso de la Opción 3 es multiplicado por el factor $(1-\delta)^3$. Muestra una supresión importante en el límite $m^2\to M^2$.
\end{enumerate}

Sabemos por la sección \ref{subsec:QCDLag} que el valor del acoplamiento fuerte varía con la escala de energía del proceso. Este efecto no puede ser despreciado en la lluvia de partones, donde los decaimientos sucesivos hacen que las virtualidades decrezcan. Hemos supuesto un valor fijo de $\as$ para hacer que las fórmulas luzcan fáciles, pero la dependencia del acoplamiento con la escala de energía está fijada en el algoritmo.

Varias amplitudes interfieren en la emisión de gluones, y ciertos cálculos han mostrado que $\pTse$ es la variable óptima para el acoplamiento $\as$. Estos argumentos no necesariamente se pueden aplicar a $\g\to\q\qbar$. En su lugar, ha sido sugerido que la $m^2$ del gluón virtual que decae en $\Q\Qbar$ es una mejor variable. Tal alternativa es explorada multiplicando por un factor de $\log(\pTse/\Lambda^2)/\log(m^2/\Lambda^2)$, como puede ser entendido de la ecuación (\ref{eq:alphastrong}).

Las opciones 1-4 tienen sus equivalentes 5-8 escogiendo $\as(m^2)$. En \textsc{Pythia}, las opciones pueden ser usadas asignando la variable \verb|TimeShower:weightGluonToQuark|.

