\chapter{Conclusiones}
\label{sec:summary}

La tasa de producción $\g_{\b\bbar}$ fue simulada y comparada con resultados experimentales. Los resultados están de acuerdo con la opción por defecto de \textsc{Pythia} y con la opción 4 dentro de los errores experimentales. Curiosamente, el aumento producido por la opción 4 en la región de umbral de masa es casi exactamente cancelado por la supresión a masas altas, lo que conllevó a una tasa cercana a la dada por la opción por defecto. La opción 2 da un resultado dentro de dos desviaciones estándar comparado con los datos experimentales, mientras que la opción 3 no parece reproducir la tasa. Las opciones 5-8, usando $m^2$ como el argumento  del acoplo fuerte, no afecta sensiblemente la tasa (alrededor de 5\% de diferencia para cada opción).

Existe un limitado conjunto de datos disponibles para el estudio de quarks pesados en colisionadores leptónicos. Futuras mediciones de la producción de quarks pesados como función de la masa invariante del par pudiera esclarecer cuál de las opciones es la más adecuada, o la necesidad de una nueva.

El estudio no permite concluir definitivamente sobre la base del espectro de energía de los mesones $\D^{*\pm}$. La opción por defecto muestra una deficiencia a bajas energías que pudiera ser corregida por el aumento dado por las opciones alternativas, particularmente por la opción 3. Sin embargo, todas las opciones presentan un exceso a energías medianas que, de ser corrido a regiones de más bajas energías a través de, por ejemplo, un nuevo modelado del decaimiento $\B\to\D$, pudiera también reconciliar \textsc{Pythia} y los datos experimentales, sin introducir una nueva tasa de producción $\g\to\Q\Qbar$.

Para colisiones protón-protón a energías típicas del LHC, la abertura angular azimutal, las rapideces relativas y las distancias $R$ de pares de mesones $\B$ fueron simuladas. Las contribuciones de los mecanismos de producción a las cantidades mencionadas fueron mostradas. La variación de los observables usando las diferentes opciones de \textsc{Pythia} también fue estudiada y mostrada en los gráficos. Usando cortes inferiores para el momentum transveso para la generación de eventos y para los mesones $\B$ analizados, los eventos seleccionados fueron alrededor de $1.5$ \% de los generados.

Futuros estudios pudieran empezar comparando los resultados simulados de las cuatro opciones con los datos. Existen resultados experimentales y análisis para los experimentos Tevatron y LHC, basados en diferentes mecanismos de reconstrucción de eventos. La dependencia de las correlaciones y tasas de producción  implementando otras PDFs pueden ser también exploradas en este contexto.

